\section{User Guide for Using the Interface}
\subsection{Transmitter Interface}
\begin{figure}[ht]
  \centering
    \includegraphics[height=8cm]{manual.png}
  \caption{Function of each elements of Transmitter Interface}
  \label{fig:manual}
\end{figure}
We start by fixing the parameters in Control Panel :
\begin{enumerate}
	\item Select the transmission method
	\item Select the number of antennas(for example if we selected '1-4', it means 1 transmitter antenna and 4 receiver antennas)
	\item Select the loop types. By default,it is closed loop
	\item If we selected closed loop, we can choose the Desired Bit Error Rate(this option is disabled if open loop is selected)
	\item Enter the Desired Transmission Time value (optional) 
\end{enumerate}
\clearpage
\par Next, we proceed to the Channel Panel. Here, we will select the channel for file transmission. The carrier frequency of the selected channel will be displayed at the bottom.

\par Then, we proceed to the Data Panel. To start with, we select the stream data rate and browse the file that we want to use. Finally click on the Start button to begin the transmission(make sure to do the 1st step in the Receiver Interface before clicking).

\subsection{Receiver Interface}

\begin{figure}[ht]
  \centering
    \includegraphics[width=14cm]{manual2.png}
  \caption{Function of each elements of Receiver Interface}
  \label{fig:manual2}
\end{figure}

The steps are as the following :
\begin{enumerate}
	\item We start by fixing the same value as we did on Transmitter Interface for parameters in Control and Channel Panel. 
	\item Then, we proceed to Data Panel where we can acquire the data that will be displayed on Display Panel and Value Panel.
\end{enumerate}



 
