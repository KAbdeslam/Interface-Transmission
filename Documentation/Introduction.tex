\section{Introduction}

\par In the context of a radio test platform for the transmission of multimedia content in a difficult environment, the SYSCOM team of XLIM-SIC laboratory has developed a transmission system based on MIMO-OFDM technology that also adopted the latest standards of high-speed wireless communication. The next step is to create a program that make it easier for the user that has no programming knowledge. To achieve this objective, it is necessary to create an interface using Matlab that is unified with the transmission system coded with C++ language. 

\par \vspace{0.25cm}The importance of this project is not only to learn programming a Graphical User Interface(GUI) using Matlab, but it is also to learn about the MIMO-OFDM technology and the 802.11a standard and learn to manage the project using the Scrum Agile method. As MIMO-OFDM technology is widely used in the wireless network, it will be beneficial for us to know how it works. Furthermore, the Scrum Agile method is widely used in the programming domain so it will an added value in our skill. 

\par \vspace{0.25cm}It also imperative to state that there are also some limitation from the beginning. As our client want us to focus on optimizing our interface, we are not provided with the code for the transmission and the devices used to simulate the radio test. This caused a major setback of what we have planned before. Rather than doing the analysis on the test simulation, we did the analysis on our design. It's difficult to analyze the aesthetic of our interface but it is also a new experience for us. 

\par \vspace{0.25cm}As our client demands to use a specific software to program the interface, we need to adapt to the limitation of Matlab in creating GUI. Matlab is specialized in integrating GUI that has complex functions rather than the visual look of GUI. The transmission code that they gave us simplify our task as it is unnecessary to code for file transmission. But it's also became a problem as we cannot fully integrate the code with the parameters that we have envisioned at the beginning. 
